
In this work, we argue that there are two distinct types of information that need to be encoded in the messages $m_{j \to i}$. The first we refer to as \textit{sensory information}, which represents attributes or features of individual objects. The second is \textit{relational information} about the relationship between the sender and receiver along various dimensions.
% \footnote{This terminology of ``sensory'' and ``relational'' is borrowed from the cognitive neuroscience literature~\citep[e.g.,][]{holyoak2012analogy}. The generic term ``object'' is used to refer to elements of the input, which is typically a sequence or set. For example, objects might be the tokens in the context of text or the patches of an image in the context of vision.}. 
The Transformer architecture captures the transmission of sensory information, but does not explicitly support the transmission of relational information.

\aawarning*{to-add}{Modeling sequences requires reasoning about two types of information---1) the attributes or features of each element in the sequence, and 2) the relationships between these elements (i.e., comparisons across different attributes or feature dimensions). We refer to the first type of computation as \textit{sensory} computation and the second as \textit{relational}. [Modern neural architectures for sequence modeling, like Transformers, are [highly effective] at modeling sensory computation, e.g., owing to neural attention supporting routing of sensory information between elements in the sequence as needed for the downstream task (learned input-dependent) .] However, beyond this, the Transformer lacks an ability to directly route relational information between elements in the sequence, making it difficult to capture this information in the latent representations. [It is [essential/useful] to compose both types of processing mechanisms to enable rich representation learning.]/[Our goal is to design a neural architecture imbued with both types of computational mechanisms, to enable rich representation learning supporting a wide range of tasks and data modalities.] }

\aanote{motivate why relational learning is important. cogsci motivation. abstraction/generalization? }

In this paper, we propose a novel type of attention mechanism that explicitly encodes learned relations between the sender and receiver. For this attention mechanism, the message from the sender object to the receiver object is a set of relations between them, which can be expressed as $m_{j \to i} = r(x_i, x_j)$, with the relations $r(\cdot,\cdot)$ computed through inner products of feature maps which represents comparisons across various attributes or feature dimensions . By combining this with the standard attention mechanism of Transformers, we obtain a model that explicitly processes both sensory and relational information by stacking the two types of attention heads with $m_{j \to i} = (\phi_v(x_j), r(x_i, x_j))$. This \textit{Dual Attention} architecture disentangles the two types of information in the aggregation phase of attention, while making both types of information available during the information processing stage.




\aawarning*{Broad outline}{
\begin{enumerate}
  \item broad goal: unified architectures capable of performing variety of tasks across different modalities. (i.e., compatible with different paradigms [classification, seq2seq, autoregressive, etc.] while possessing inductive biases and computational mechanisms for )
  \item Tension between generality and inductive biases
  \item Relational learning: motivate importance 
  \begin{enumerate}
    \item Motivate importance of relational learning. Perhaps with some CogSci background. Say that this is an area of active researc.
    \item give some background on previous work, and explain message that Transformers struggle with relational learning and reasoning. State explicitly: for example, these works demonstrates/construct (synthetic) tasks which Transformer models fail to learn in a data-efficient manner and propose (alternative) architectures with relational inductive which enable improved efficiency and greater effectiveness on this class of tasks. (Maybe give high-level description of what these tasks involve)
  \end{enumerate}
  \item State our goal
  \item Give high-level motivation of our proposed solution by describing message-passing interpretation of attention
  \item Close intro by motivating goals of remainder of paper (and summarizing contributions? perhaps redundant given description of goals...)
\end{enumerate}
}

\aanote{Add related work section?}


Neural attention provides a promising candidate for the basis of a versatile general-purpose architecture for machine learning 

A promising candidate

Power of relational learning~\citep{goyal2022inductive}.

1) Transformers, through neural attention mechanisms for routing information, (which bears resemblance to systems studied in cognitive neuroscience) provides a promising candidate for a versatile, flexible, robust, and adaptable unified machine learning architecture.

Relational learning is especially important to higher-level ``system 2'' conscious reasoning processes, which are believed to be crucial for systematic (out-of-distribution) generalization and compositionality (etc. what other descriptors).

Hypothesis: human \& animal intelligence could be explained by a few fundamental principles.

focus on sample-efficiency and generalization...

No-free lunch theorem: no single architecture can support all possible tasks (theoretically); practically, modern deep learning techniques owe their success to certain useful inductive biases (e.g., optimization-based ones such as implicit regularization)

The promising architecture~\citep{vaswani2017attention} is a promising candidate for this goal, owing to its ability to process general inputs and outputs encoded as sequences or sets, and the versatility of the Attention+MLP design.

\aanote{Formulate goal as uncovering the fundamental general computational mechanisms which would support a wide range of tasks...}

The Transformer architecture~\citep{vaswani2017attention} is a promising candidate for this goal.  as it is is able to process general inputs and produce general outputs encoded as as sequences or sets of objects, and possess computational mechanisms neural attention

The generality of the input and output formats for sequence models such as Transformers~\citep{vaswani2017attention} makes them promising candidates for this goal. In particular, the Transformer architecture's \aawarning*{TODO---revise}{[Attention+MLP design]} provides highly-general computational mechanisms that have found success across several important domains, notably including language~\citep{devlinBERTPretrainingDeep2019,radford2019language} and visual processing~\citep{dosovitskiyImageWorth16x162020}.

However, a tension exists between the goal of having a general architecture and the ability to support inductive biases that may be favorable for certain types of tasks. Recent research has shown that the standard Transformer architecture lacks the ability to efficiently learn and represent relational information, leading to several proposals for alternative architectures to encode inductive biases for relational learning \citep{santoroSimpleNeuralNetwork2017,santoroRelationalRecurrentNeural2018,shanahanExplicitlyRelationalNeurala,webbEmergentSymbolsBinding2021,webbRelationalBottleneckInductive2024,kergNeuralArchitectureInductive2022,altabaa2024abstractors,altabaaLearningHierarchicalRelational2024}. These relational architectures, on the other hand, fail to provide the generality required to handle more general learning tasks. In this paper, we present an extension of the Transformer architecture that preserves the generality of the architecture while integrating inductive biases for processing relational information between objects.

The Transformer can be understood as an instantiation of a broader computational paradigm implementing a neural message-passing algorithm that consists of iterative information retrieval followed by local processing. To process a sequence of objects $x_1,\ldots, x_n$, this takes the general form
\begin{equation}\label{eq:intro_message_passing}
  \begin{split}
    x_i &\gets \mathrm{Aggregate}\bigparen{x_i, \set{m_{j \to i}}_{j=1}^{n}} \\
    x_i &\gets \mathrm{Process}(x_i)
  \end{split}
\end{equation}
In the case of Transformers, the self-attention mechanism can be seen as sending messages from object $j$ to object $i$ that are encodings of the sender's features, with the message from sender $j$ to receiver $i$ given by $m_{j \to i} = \phi_v(x_j)$. These messages are then aggregated according to some selection criterion based on the receiver's features,  typically given by the softmax attention scores. In this work, we argue that there are two distinct types of information that need to be encoded in the messages $m_{j \to i}$. The first we refer to as \textit{sensory information}, which represents attributes or features of individual objects. The second is \textit{relational information} about the relationship between the sender and receiver along various dimensions.
% \footnote{This terminology of ``sensory'' and ``relational'' is borrowed from the cognitive neuroscience literature~\citep[e.g.,][]{holyoak2012analogy}. The generic term ``object'' is used to refer to elements of the input, which is typically a sequence or set. For example, objects might be the tokens in the context of text or the patches of an image in the context of vision.}. 
The Transformer architecture captures the transmission of sensory information, but does not explicitly support the transmission of relational information.

\aawarning*{to-add}{Modeling sequences requires reasoning about two types of information---1) the attributes or features of each element in the sequence, and 2) the relationships between these elements (i.e., comparisons across different attributes or feature dimensions). We refer to the first type of computation as \textit{sensory} computation and the second as \textit{relational}. [Modern neural architectures for sequence modeling, like Transformers, are [highly effective] at modeling sensory computation, e.g., owing to neural attention supporting routing of sensory information between elements in the sequence as needed for the downstream task (learned input-dependent) .] However, beyond this, the Transformer lacks an ability to directly route relational information between elements in the sequence, making it difficult to capture this information in the latent representations. [It is [essential/useful] to compose both types of processing mechanisms to enable rich representation learning.]/[Our goal is to design a neural architecture imbued with both types of computational mechanisms, to enable rich representation learning supporting a wide range of tasks and data modalities.] }

\aanote{motivate why relational learning is important. cogsci motivation. abstraction/generalization? }

In this paper, we propose a novel type of attention mechanism that explicitly encodes learned relations between the sender and receiver. For this attention mechanism, the message from the sender object to the receiver object is a set of relations between them, which can be expressed as $m_{j \to i} = r(x_i, x_j)$, with the relations $r(\cdot,\cdot)$ computed through inner products of feature maps. By combining this with the standard attention mechanism of Transformers, we obtain a model that explicitly processes both sensory and relational information by stacking the two types of attention heads with $m_{j \to i} = (\phi_v(x_j), r(x_i, x_j))$. This \textit{Dual Attention} architecture disentangles the two types of information in the aggregation phase of attention, while making both types of information available during the information processing stage.

% \jlnote{More about Abstractors here, how we're building on it?}
% \aanote{maybe footnote directing reader to~\Cref{sec:appdx_abstrator}?}

% \aanote{note: for now, removed long paragraph summarizing experimental results}
% A series of four experiments across a range of data types demonstrates how the Dual Attention Transformer architecture preserves the benefits of standard Transformers while leading to greater efficiency of learning by combining relational information and sensory information. To begin, we use the Relational Games dataset which tests a model's ability to identify a particular visual relationship among a series of objects. We use a type of Vision Transformer architecture~\citep{dosovitskiyImageWorth16x162020}, and evaluate learning curves of the models, finding that the Dual Attention architecture is significantly more sample efficient compared to a standard Transformer. Next, we evaluate the Dual Attention architecture on a set of mathematical problem-solving tasks based on the benchmark contributed by~\citet{saxtonAnalyzingMathematicalReasoning2019}, where the tasks range \aawarning*{typo}{from to} differentiating functions, finding that the Dual Attention models learn faster and reach higher accuracies compared to a standard Transformer. We also evaluate the new architecture on autoregressive language modeling. Using the Tiny Stories dataset of~\citet{eldanTinyStoriesHowSmall2023} to train models \aawarning*{actually, our models are not on that scale, they're significantly smaller.}{on the roughly the scale of GPT-2}, we fix the total number of attention heads, and compare a Transformer with only standard self-attention heads to Dual Attention models with a mix of self-attention and relational attention heads, finding that the Dual Attention models achieve a smaller validation loss for the same total number of attention heads. Finally, we again use a Vision Transformer for classification of ImageNet images \citep{dosovitskiyImageWorth16x162020}, and find improved efficiency of learning. This suggests that relational processing is important in processing visual scenes, matching the intuition that parsing a visual scene requires reasoning about the visual relations between different objects or parts of the scene. Together, the experiments show that the Dual Attention Transformer is effective across a range of tasks and data types.

The contributions in this paper are summarized as follows:
\begin{enumerate}
  \item We introduce a new \textit{relational attention} mechanism that disentangles sensory information from relational information. While standard self-attention models retrieval of sensory information, relational attention models retrieval of relational information.
  \item We introduce a generalized Transformer architecture that integrates sensory and relational information through \textit{Dual Attention}---a form of multi-head attention with two distinct types of attention heads. Standard self-attention heads encode sensory information while relational attention heads encode relational information.
  \item \aawarning*{update / revise this sentence}{We carry out an thorough set of experiments, showing that the \textit{Dual Attention Transformer} outperforms standard Transformers in terms of data efficiency. Our experiments range across several tasks and data modalities, including visual relational reasoning, symbolic mathematical reasoning, language modeling, and image recognition.}
\end{enumerate}

% \aanote{note: for now, commented out paragraph about meaning of the terms ``sensory'', ``relational'', and ``objects''. TODO--find a place for it? e.g., footnote? or put back in intro?}
% Note that throughout the paper, the term ``sensory'' is used as shorthand to refer to the features and attributes of an individual object. For example, in a language task, this could be the components of a word embedding, and in a vision task, this could be the pixel values of a patch of an image or the feature maps obtained using a CNN. We use the term ``relational'' to refer to information about the relationship between the features of two objects. For example, in a language task, this could be the grammatical, syntactic, or semantic relation between two words. In a vision task, this could be a representation of similarity across different visual attributes such as the color, texture or semantic category of two objects. In the proposed learning architecture, the sensory information and relations are both learned, rather than being specified in advance of processing data for a given task.  The generic term ``object'' is used to refer to elements of the input, which is typically a sequence or set. For example, objects might be the tokens in the context of text or the patches of an image in the context of vision.

% \aawarning{TODO -- add sentence on potential congitive science connection? ``we borrow this language from recent work in the cognitive science literature, which argues that ...''}

% \aanote{\tiny{[An emerging idea in cognitive science] is that there essentially exists two distinct types of information when reasoning about collections of objects: ``sensory'' information concerning the attributes of individual objects, and ``relational'' information about the relationships between objects. It is contended that the first is associated with procedural knowledge whereas the second is associated abstraction. The standard attention mechanism of Transformers naturally captures retrieval of sensory information, but does not explicitly support reasoning about relational information.}}

\aawarning*{to add... intent should be to prime reader so that they expect what's to come in the methods and experiments sections.}{
We focus our [investigation] on the following guiding questions:
\begin{itemize}
  \item How can sensory and relational computation be supported in a single integrated architecture?
  \item Can an architecture with both sensory and relational computational mechanisms retain the benefits of relational inductive biases in relational tasks (e.g., synthetic benchmarks that evaluate relational learning)?
  \item Do relational computational mechanisms confer performance benefits in complex real-world tasks such as visual processing or language modeling? How does performance scale with respect model size and data size in those settings?
\end{itemize}
}

\aawarning*{to incorporate...}{One message of the motivation for the paper could be the following (alternatively, could be conveyed in a related work section): previous work on ``relational learning'' has been limited to simple tasks involving low-dimensional synthetic sensory inputs (e.g., the relational games benchmark considered in~\Cref{ssec:relgames}). A crucial remaining open question in this line of research is whether relational computational mechanisms are useful for complex real-world tasks such as visual processing or language modeling.}
% \aawarning{TODO---add footnote somewhere (maybe in section 2) referring reader to~\Cref{sec:appdx_abstrator} for a discussion on the relation to~\citet{altabaa2024abstractors}?}