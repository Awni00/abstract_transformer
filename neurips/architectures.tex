\section{Orthrus: The Dual-Head Transformer Architecture}

\subsection{Dual-Head Attention}

One of the keys to the success of the Transformer architecture is the use of so-called \textit{multi-head} attention. This involves learning and computing multiple attention operations in parallel at each layer and concatenating the output. This enables learning multiple useful criteria for routing information between objects. However, all these attention heads share the same inductive bias, focusing on sensory information about individual objects. Intuitively, this leads to \aanote*{is this phrasing good?}{diminishing returns} where adding more heads does not result in improved performance since all heads perform the same type of computation. In particular, a key type of inductive bias that a standard Transformer lacks is an inductive bias for processing \textit{relational} information between objects.

We posit that sensory and relational information are the two primary types of information that are of relevance when processing sequences or collections of objects. This idea has some support \aanote*{motivations/support/roots... The congitive neuroscience literature contains work in support of this hypothesis regarding how the human mind works.}{support} in cognitive (neuro)science~\citep{citation}. In this paper, we explore the effects of augmenting a Transformer with a specialized attention operation with relational inductive biases. We propose a type of multi-head attention with two distinct types attention heads: standard self-attention, and relational attention. Our hypothesis is that by having both kinds of computations available to the model, it can learn to use both and select between them depending on the current task/context/etc.

\Cref{alg:dual_head_attn} describes the proposed operation. The number of self-attention heads $\nhsa$ and number of relational attention heads $\nhra$ are hyperparameters. The self-attention heads attend to and retrieve sensory information on the features of individual objects in the context while the relational attention heads attend to and retrieve relational information. The $n_h = \nhsa + \nhra$ heads are then concated to produce the output. The result is a representation of contextual information with disentangled sensory and relational components.

We note that a symmetry inductive bias can be injected into the relations $\bm{r}_{ij}$ by imposing that $W_{q}^{\rel} = W_k^{\rel}$. We refer the reader to~\Cref{sec:appendix_implementation} for further discussion and implementation details.

\aanote{should we call it ``dual-head'' or ``dual head-type'' or ``dual-type''}

\begin{algorithm}[ht!]
	\caption{Dual-Head Attention}\label{alg:dual_head_attn}
	\SetKwInput{Hyperparameters}{Hyperparameters}
	\SetKwInput{Input}{Input}
	\SetKwInOut{Output}{Output}
    \Hyperparameters{$\nhsa, \nhra$, symbol assignment mechanism, (optional: \texttt{symmetric\_RA = False}, $d_r = \nhra$, ...)}
	\Input{$\x = (x_1, \ldots, x_n) \in \reals^{n \times d}$ }

    \vspace{1em}

    \texttt{Compute self-attention heads}
    \begin{align*}
        \bm{\alpha}^{(h)} &\gets \Softmax\bigparen{{(\x \, W_q^h)} {(\x \, W_k^h)}^\intercal}, \qquad && h \in [\nhsa] \\
        e_i^{(h)} &\gets \sum_{j} \alpha_{ij}^{(h)} x_j \, W_v^h, \qquad && i \in [n], h \in [\nhsa]\\
        e_i &\gets \concat\bigparen{e_i^{(1)}, \ldots, e_i^{(\nhsa)}} \, W_o^{sa}, && i \in [n]
    \end{align*}

    \texttt{Assign symbols:} $\s = (s_1, \ldots, s_n) \gets \SymbolRetriever(\x;\,\Slib)$

    \texttt{Compute relational attention heads}
    \begin{align*}
        \bm{\alpha}^{(h)} &\gets \Softmax\bigparen{{(\x \, \Wqattnn{h})} {(\x \, \Wkattnn{h})}^\intercal}, \qquad && h \in [\nhra] \\
        \bm{r}_{ij} &\gets \bigparen{\iprod{x_i \, \Wqrell{\ell}}{x_j \, \Wkrell{\ell}}}_{\ell \in [d_r]} \qquad && i,j \in [n] \\
        a_i^{(h)} &\gets \sum_{j} \alpha_{ij}^{(h)} \bigparen{\bm{r}_{ij}\,W_r^h + s_j \, W_s^{h}}, \qquad && i \in [n],\, h \in [\nhra]\\
        a_i &\gets \concat\bigparen{a_i^{(1)}, \ldots, a_i^{(\nhra)}} W_o^{ra}, && i \in [n]
    \end{align*}

    \Output{$\bigparen{\concat(e_i, a_i)}_{i=1}^{n}$}

\end{algorithm}

\textbf{Attention Masks \& Causality.} Any type of attention mask (e.g., causal mask for autoregressive language modeling) can be implemented in relational attention in the same way as for standard self-attention (i.e., mask is added to $\alpha_{ij}^h$ pre-softmax).

\textbf{Positional Encoding.} There exists different methods in the literature on encoding positional information in the Transformer architecture. For example,~\citet{vaswani2017attention} propose adding positional embeddings to the input,~\citet{shawSelfAttentionRelativePosition2018b} propose adding relative-positional embeddings at each attention operation, and~\citet{suRoFormerEnhancedTransformer2023} propose rotary positional embeddings (RoPE) which apply a position-dependent map to the queries and keys pre-softmax. These methods are compatible with dual-head attention and are configurable options in our public implementation.

\textbf{Computational complexity.} The computational complexity of dual-head attention scales the same as self-attention with a $O(n^2)$ dependence on sequence length. Like standard attention, relational attention can be computed in parallel via efficient matrix multiplication operations.

\aanote[inline,nomargin]{To increase parameter efficiency, we suggest that the attention query/key projections can be shared across self-attention and relational attention heads, with the intuition that the same selection criteria would be useful for retrieving either object-level or relational information. We leave testing this idea for future work. [Put this in appendix together with disucssing hyperparameters, learnable parameters, design choices, etc.]}

\subsection{Model Architecture}

A standard Transformer implements procedure of iterative information retrieval (attention) followed by local processing (MLP). We define our dual-head Transformer in the same way, except that self-attention is replaced by dual-head attention.~\Cref{alg:dh_encoder,alg:dh_decoder} defines an encoder and decoder block with dual-head attention. In our dual-head Transformer architecture, these blocks are simply stacked in layers, as in standard Transformers.

\begin{remark}
    In our implementation, the symbol library $\Slib$ is shared across layers to reduce the number of parameters. The interpretation is that although a particular symbol may have different ``semantic meanings'' in different layers, since its role is to merely act as an abstract code, it can be remapped at each layer (e.g., a variable can be reassigned to point to a different object). Moreover, $\Slib$ can be randomly initialized and frozen to further reduce the number of trainable parameters. This is because a good set of symbols needs only to be well-separated and random gaussian codes/vectors are approximately orthogonal in high-dimensions~\citep{needcitation?citationonrandomcodes?}.
\end{remark}

\begin{figure}[ht]
    \begin{minipage}{0.45\textwidth}
        \begin{algorithm}[H]
            \caption{Dual-Head Encoder Block}\label{alg:dh_encoder}
            \SetKwInOut{Input}{Input}
            % \SetKwInOut{Hyperparameters}{Hyperparameters}
            % \Hyperparameters{$\nhsa, \nhra$, symbol assignment mechanism, (optional: \texttt{symmetric\_RA = False}, $d_r = \nhra$, ...)}
            \Input{$\x \in \reals^{n \times d}$}

            $\x \gets \mathrm{Norm}(\x + \mathrm{DualHeadAttn}(\x))$

            $\x \gets \mathrm{Norm}(\x + \MLP(\x))$

            $\hphantom{\x \gets \mathrm{Norm}(\x + \MLP(\x))}$

            \textbf{Output:} $\x$
        \end{algorithm}
    \end{minipage}
    \hfill
    \begin{minipage}{0.45\textwidth}
        \begin{algorithm}[H]
            \caption{Dual-Head Decoder Block}\label{alg:dh_decoder}
            \SetKwInOut{Input}{Input}
            \SetKwInOut{Output}{Output}
            \Input{$\x, \y \in \reals^{n \times d}$}

            $\x \gets \mathrm{Norm}(\x + \mathrm{DualHeadAttn}(\x))$

            $\x \gets \mathrm{Norm}(\x + \mathrm{CrossAttn}(\x, \y))$

            $\x \gets \mathrm{Norm}(\x + \MLP(\x))$

            \Output{$\x$}
        \end{algorithm}
    \end{minipage}
\end{figure}

\aanote{mention interpretation of composing blocks of relational attention: learning representations of higher-order relations}
An encoder-decoder architecture with causal dual-head attention in the decoder can be applied to sequence-to-sequence tasks, as in the original Transformer paper~\citep{vaswani2017attention}. An encoder-only architecture can be used for a BERT-style language embedding model~\citep{devlinBERTPretrainingDeep2019} or a Vision Transformer-style vision model~\citep{dosovitskiyImageWorth16x162020}. A decoder-only architecture with causal dual-head attention can be used for autoregressive language modeling\footnote{``Decoder-only'' is the commonly used term for this type of architecture~\citep{radfordImprovingLanguageUnderstanding2018}, but it can also be viewed as an encoder-only architecture with causal attention}.

We nickname models with this ``dual head-type'' architecture \aanote*{any other options for 2-headed mythological creatures? Orthrus may be slightly hard to pronounce? (for silly people like me)}{Orthrus}, after the two-headed dog in greek mythology. We note that a standard Transformer is a special case of this architecture where $\nhsa = n_h, \nhra = 0$.

\aanote{Mention that our publicly available code exposes all of these as easy-to-choose hyperparameters in a single implementation (maybe also put on Hugging face, etc. make it very easy for people to use. but choose reasonable defaults so people don't need to make choices for too many hyperparameters)}

\aanote{mention pre-norm/post-norm, LayerNorm vs RMSNorm, etc.}